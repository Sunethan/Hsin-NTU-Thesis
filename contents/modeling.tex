\chapter{TCAD 模擬}\label{c:modeling}
\section{數值模擬原理}
\section{物理模型設定}
\subsection{鋅擴散模型}
\subsection{連續方程式}
Boltzmann transport equation is a continuity equation for $f(r,k,t)$ (6.730 Physics for Solid State Applications Lecture 32: Introduction to Boltzmann Transport).
\subsection{能帶穿隧模型}
\subsection{遷移率模型}
\subsection{SRH復合模型}
\subsubsection{電場效應}
\subsubsection{濃度效應}
\subsection{撞擊游離模型}
\begin{quote}
For example, if we are considering the bound states of an impurity centre in an $n$-type semiconductor we may obtain a good description of these states using only wave functions of the conduction band provided the ionization energy $W_I$ is small compared with the forbidden energy gap $\Delta E$. This will be true for the so-called 'shallow' impurity levels given by group V elements in silicon or germanium, but not for the deep-lying impurity levels having energy levels near the cnetre of the conduction band. For the latter the differential equation will be invalid in any case since $V(\mathbf{r})$ is likely to vary too rapidly to enable us to use the approximate method of calculating $V_{jj^\prime}$. In this case, the set of linear algebraic equations must be used and not the wave equation for $F(\mathbf{r})$. In general, the solution of the problem becomes very difficult and has not been carried out in many cases of interest.~\cite{smith1963wave}
\end{quote}
\section{參數萃取模擬實驗}
\subsection{穿隧模型參數之萃取}
\subsection{縱向摻質濃度之萃取}
\section{改善模擬收斂性之方法:外接電阻法}